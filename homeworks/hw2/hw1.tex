%=======================02-713 LaTeX template, following the 15-210 template==================
%
% You don't need to use LaTeX or this template, but you must turn your homework in as
% a typeset PDF somehow.
%
% How to use:
%    1. Update your information in section "A" below
%    2. Write your answers in section "B" below. Precede answers for all 
%       parts of a question with the command "\question{n}{desc}" where n is
%       the question number and "desc" is a short, one-line description of 
%       the problem. There is no need to restate the problem.
%    3. If a question has multiple parts, precede the answer to part x with the
%       command "\part{x}".
%    4. If a problem asks you to design an algorithm, use the commands
%       \algorithm, \correctness, \runtime to precede your discussion of the 
%       description of the algorithm, its correctness, and its running time, respectively.
%    5. You can include graphics by using the command \includegraphics{FILENAME}
%    6. If you want to use code, use \begin{lstlisting} after making modifications to the lstset.
\documentclass[11pt]{article}
\usepackage{amsmath,amssymb,amsthm}
\usepackage{graphicx}
\usepackage[margin=1in]{geometry}
\usepackage{fancyhdr}
\usepackage{listings}
\usepackage{color}
\usepackage{enumitem}
\setlength{\parindent}{0pt}
\setlength{\parskip}{5pt plus 1pt}
\setlength{\headheight}{13.6pt}
\newcommand\question[2]{\vspace{.25in}\hrule\textbf{#1: #2}\vspace{.5em}\hrule\vspace{.10in}}
\renewcommand\part[1]{\vspace{.10in}\textbf{(#1)}\par}
\newcommand\algorithm{\vspace{.10in}\textbf{Algorithm: }}
\newcommand\correctness{\vspace{.10in}\textbf{Correctness: }}
\newcommand\runtime{\vspace{.10in}\textbf{Running time: }}
\definecolor{javared}{rgb}{0.6,0,0} % for strings
\definecolor{javagreen}{rgb}{0.25,0.5,0.35} % comments
\definecolor{javapurple}{rgb}{0.5,0,0.35} % keywords
\definecolor{javadocblue}{rgb}{0.25,0.35,0.75} % javadoc
\lstset{language=C,
basicstyle=\ttfamily,
keywordstyle=\color{javapurple}\bfseries,
stringstyle=\color{javared},
commentstyle=\color{javagreen},
morecomment=[s][\color{javadocblue}]{/**}{*/},
numbers=left,
numberstyle=\tiny\color{black},
stepnumber=1,
numbersep=10pt,
tabsize=4,
showspaces=false,
showstringspaces=false}
\pagestyle{fancyplain}
\lhead{\textbf{\NAME}}
\chead{\textbf{{\COURSE} HW\HWNUM}}
\rhead{\today}
\begin{document}
%Section A==============Change the values below to match your information==================
\newcommand\NAME{Eric Altenburg}  % your name
\newcommand\COURSE{CS 492}
\newcommand\HWNUM{2}              % the homework number
%Section B==============Put your answers to the questions below here=======================

% no need to restate the problem --- the graders know which problem is which,
% but replacing "The First Problem" with a short phrase will help you remember
% which problem this is when you read over your homeworks to study.

\begin{center}
	\textit{\textbf{Pledge:} I pledge my honor that I have abided by the Stevens Honor System.} - \textbf{\NAME}
\end{center}


\question{1}{Does the busy waiting solution using the turn variable (Fig 2-23 in the book) solves the mutual exclusion problem when two processes are running on a shared-memory multiprocessor, that is, two CPUs sharing a common memory?}
	Yes, the busy waiting solution using the turn variable solves the mutual exclusion problem when two processors are running on a shared-memory multiprocessor. From the figure, the "turn" variable is used to control the critical region between the two processes, and in this case, since it is on a multiprocessor, the two processes are running on different CPUs; however, the "turn" variable should now be in the shared-memory.

\question{2}{Five batch jobs. A through E, arrive at a computer center at almost the same time. They have estimated running times of 10, 6, 2, 4, and 8 minutes. Their (externally determined) priorities are 3, 5, 2, 1, and 4, respectively, with 5 being the highest priority. For each of the following scheduling algorithms, determine the mean process turnaround time. Ignore process switching overhead.
\begin{enumerate}[label=\alph*]
	\item Round robin.
	\item Priority scheduling.
	\item First-come, first-served (run in order 10, 6, 2, 4, 8).
	\item Shortest job first. 
\end{enumerate}
For (a), assume that the system is multiprogrammed, and that each job gets its fair share of the CPU. For (b) through (d), assume that only one job at a time runs, until it finishes. All jobs are completely CPU bound.}
	
All batch jobs have been assigned a letter (a-e) in the order they were stated.\par
	\part{a}
		\begin{tabular}{|c|c|c|c|c|c|c|c|c|c|c|c|c|c|c|}
			\hline
			1 & 2 & 3 & 4 & 5 & 6 & 7 & 8 & 9 & 10 & 11 & 12 & 13 & 14 & 15\\
			\hline
			A & B & C & D & E & A & B & C & D & E & A & B & D & E & A\\
			\hline
		\end{tabular}\par
		\begin{tabular}{|c|c|c|c|c|c|c|c|c|c|c|c|c|c|c|}
			\hline
			16 & 17 & 18 & 19 & 20 & 21 & 22 & 23 & 24 & 25 & 26 & 27 & 28 & 29 & 30\\
			\hline
			B & D & E & A & B & E & A & B & E &  A& E & A & E & A & A\\
			\hline
		\end{tabular}\par
		The turnaround times in minutes for each job are as follows: \par
		A 30\\
		B 23\\
		C 8\\
		D 17\\
		E 28\par
		$\text{Mean Process Turnaround Time} = \frac{30 + 23+ 8+ 17+ 28}{5}=21.2 \text{ minutes}$

	\part{b}
		\begin{tabular}{|c|c|c|c|c|}
			\hline
			1-6 & 7-14 & 15-24 & 25-26 & 27-30\\
			\hline
			B & E & A & C & D\\
			\hline
		\end{tabular}\par
		The turnaround times in minutes for each job are as follows:\par
		A 24\\
		B 6\\
		C 26\\
		D 30\\
		E 14\par
		$\text{Mean Process Turnaround Time} = \frac{24 + 6 + 26 + 30 + 14}{5} = 20 \text{ minutes}$

	\part{c}
		\begin{tabular}{|c|c|c|c|c|}
			\hline
			1-10 & 11-16 & 17-18 & 19-22 & 23-30\\
			\hline
			A & B & C & D & E\\
			\hline
		\end{tabular}\par
		The turnaround times in minutes for each job are as follows:\par
		A 10\\
		B 16\\
		C 18\\
		D 22\\
		E 30\par
		$\text{Mean Process Turnaround Time} = \frac{10 + 16 + 18 + 22 + 30}{5} = 19.2 \text{ minutes}$

	\part{d}
		\begin{tabular}{|c|c|c|c|c|}
			\hline
			1-2 & 3-6 & 7-12 & 13-20 & 21-30\\
			\hline
			C & D & B & E & A\\
			\hline
		\end{tabular}\par
		The turnaround times in minutes for each job are as follows:\par
		A 30\\
		B 12\\
		C 2\\
		D 6\\
		E 20\par
		$\text{Mean Process Turnaround Time} = \frac{30 + 12 + 2 + 6 + 20}{5} = 14 \text{ minutes}$


\newpage
\question{3}{A soft real-time system has four periodic events with periods of 50, 100, 200, and 250 msec each. Suppose that the four events require 35, 20, 10, and x msec of CPU time, respectively. What is the largest value of x for which the system is schedulable?}	
	\begin{align*}
		\frac{35}{50} + \frac{20}{100} + \frac{10}{200} + \frac{x}{250} &\le 1\\
		0.95 + \frac{x}{250} &\le 1\\
		\frac{x}{250} &\le 0.05\\
		x &\le 12.5\\
		x &= 12.5 \text{ msec}\\
	\end{align*}
	The largest value x can be is 12.5 msec. 


\question{4}{Consider the following state of a system with four processes, P1, P2, P3, and P4, and five types of resources, RS1, RS2, RS3, RS4, and RS5:
\begin{center}
$C =$ 
\begin{tabular}{|c|c|c|c|c|}
	\hline
	0 & 1 & 1 & 1 & 2\\
	\hline
	0 & 1 & 0 & 1 & 0\\
	\hline
	0 & 0 & 0 & 0 & 1\\
	\hline
	2 & 1 & 0 & 0 & 0\\
	\hline
\end{tabular}
$R = $
\begin{tabular}{|c|c|c|c|c|}
	\hline
	1 & 1 & 0 & 2 & 1\\
	\hline
	0 & 1 & 0 & 2 & 1\\
	\hline
	0 & 2 & 0 & 3 & 1\\
	\hline
	0 & 2 & 1 & 1 & 0\\
	\hline
\end{tabular}
\begin{tabular}{c}
	$E=(24144)$\\
	$A=(01021)$
\end{tabular}
\end{center}
Using the deadlock detection algorithm described in Section 6.4.2, show that there is a deadlock in the system. Identify the processes that are deadlocked.}
	\begin{enumerate}
		\item Looking through the rows of R, it is clear that the 2nd row $(01021)$ equals $A=(01021)$
		\item Add the 2nd row of C (010101) to A, now $A=(02031)$
		\item Looking through the rows of R now that P$_2$ is marked, the 3rd row $(02031)$ equals $A=(02031)$
		\item Add the 3rd row of C (00001) to A, now $A=(02032)$
		\item Looking through the rows of R now that P$_2$ and P$_3$ are marked, none of the other rows in R are less than or equal to $A=(02032)$, thus the algorithm terminates. 
		\item All unmarked processes are in deadlock. Therefore, P$_1$ and P$_4$ are deadlocked.
	\end{enumerate}


\newpage
\question{5}{A system has four processes and five allocable resources. The current and maximum needs are as follows:
\begin{center}
	\begin{tabular}{cccc}
		& Allocated & Maximum & Available\\
		Process A & 1 0 2 1 1 & 1 1 2 1 2 & 0 0 x 1 1\\
		Process B & 2 0 1 1 0 & 2 2 2 1 0 & \\
		Process C & 1 1 0 1 0 & 2 1 3 1 0 & \\
		Process D & 1 1 1 1 0 & 1 1 2 2 1 & \\
	\end{tabular}
\end{center}
What is the smallest value of x for which this is a safe state? Show your calculations using the Banker's Algorithm.}

	$\text{Need Matrix} = Max - Allocation$\par
	\begin{tabular}{|c|c|c|c|c|}
		\hline
		0 & 1 & 0 & 0 & 1\\
		\hline
		0 & 2 & 1 & 0 & 0\\
		\hline
		1 & 0 & 3 & 0 & 0\\
		\hline
		0 & 0 & 1 & 1 & 1\\
		\hline
	\end{tabular}\par
	\begin{enumerate}
		\item If $x=0$, then there will be a deadlock, so 0 is not a valid value.
		\item If $x=1$, then P$_D$ can run to completion as P$_D=(00111)$ in the Need matrix is equal to $\text{Available}=(00111)$. We then add P$_D$ from the Allocated matrix to Available, now $\text{Available} = (11221)$.
		\item P$_A$ can run to completion now as P$_A=(01001)$ in the Need matrix is less than $\text{Available}=(11221)$. We then add P$_A$ from the Allocated matrix to Available, now $\text{Available}=(21432)$.
		\item P$_C$ can run to completion now as P$_C=(10300)$ in the Need matrix is less than $\text{Available}=(21432)$. We then add P$_C$ from the Allocated matrix to Available, now $\text{Available}=(32442)$.
		\item P$_B$ can run to completion now as P$_B=(02100)$ in the Need matrix is less than $\text{Available}=(32442)$.
		\item The smallest safe value of x is 1, and the safe sequence is D, A, C, B.
	\end{enumerate}
	


\end{document}